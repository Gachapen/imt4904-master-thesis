% !TEX root = ../Masters.tex
\chapter{Evaluation of DGEL}
DGEL consists of to main modules: \textit{generation} and \textit{evaluation}.
The evaluation module consists of the fitness component which evaluates who good the L-system plants are.
The remaining components are part of the generation component which generates plants, and uses the evaluation module to generate \textit{good} plants.
To answer if DGEL solves the research questions, both modules have to be evaluated.
The generation module will be evaluated in a technical manner to determine if it performs well or not.
The evaluation module will be evaluated using humans as to answer if the fitness function properly rates plants as aesthetically pleasing or not.

\section{Evaluation of the Generation Module}
\subsection{Method}
As explained in Section~\ref{sec:overview} and shown in Figure~\ref{fig:dgel}, there are two main processes involved in generating the model for the L-system plants: \textit{SA} and \textit{GE}.
The remaining steps are dependent on the model and do not create variations on the same model.
Thus SA and GE are the two processes of interests to evaluate.

% GE parameter tuning - have data
% GE vs random - have data
% SA parameter selection
% SA progress - have data, though not structured
% SA random vs random - need data, average of 160800 L-systems.
% GE SA vs GE uniform - need data, how to measure? 
% Comparing plants from different SA.

% talk about GE first...
% This is actually for comparing GE to random...
To make a fair comparison, the
GE uses a fixed number of generations and population size, resulting into a fix number of chromosome generations and modifications.
% Describe how this was found.
200 generations was used with a population size of 800.
The total number of chromosomes generated and modified combined is $population + population * generations = 800 + 800 * 200 = 160800$.

To evaluate the SA process, a list of questions were made:
\begin{itemize}
	\item Can SA and the grammar distribution be used to find portions of the paramter space that generate better L-system plants?
	\item Does SA enable GE to find better L-system plants?
	\item Can SA find multiple portions of the parameter space that each generate unique-looking plants?
\end{itemize}
To answer the first question, two methods were used.
First, SA was used as described in previous sections to find an optimized distribution, and the progress of it was plotted.
The progress plot was analyzed to see if SA made any meaningful progress.
Secondly, a comparison was made between the average fitness of L-systems randomly generated by a uniform distribution and L-systems randomly generated by an SA-optimized distribution.
The L-systems were generated in a completely random manner, not using any evolutionary approach such as GE.
% I need to actually run this experiment....
While this may answer if it is possible for an SA-optimized distribution to generate better plants, it does not answer if this always will be the case.

To answer the second question, the performance of GE with a uniform distribution was compared to the performance of GE with an SA-optimized distribution.
Both GE processes were run 11 times, each time extracting the best L-system, before taking the mean of the best L-systems and using a t-test to test if the means were different.

\subsection{Results}

\section{Evaluation of Evaluation Module}
\subsection{Method}
\subsection{Results}
