% !TEX root = ../Masters.tex
\chapter{Conclusion}

\section{Main Findings}
This thesis explored how to improve plant L-system generation, particularly with the goal of being usable for combining plants.
The idea behind this is to be able to create a virtual world where people may grow and cross-breed plants.
For people to enjoy this world, the plants should be aesthetically pleasing.
They should also be varied and so that the people may get new experiences.
Finally the plants should be modeled in a way such that they can be combined into new plants via cross-breeding.
Thus, the research question is formed: How can plants be generated that are aesthetically pleasing, varied and could
be used to create offspring similar to its parents?

Previous research on L-system generation has been focused on restricted L-system grammars and with fitness functions that aim at generating realistic plants.
Multiple L-system versions have been used, including D0L-systems, parametric D0L-systems, and parametric DIL-systems, which have then been evolved using evolutionary algorithms including GA, GP and GE.
Both automated fitness evaluation of L-systems, human evaluation and similarity evaluation have been used as selections strategies in the evolutionary algorithms.
Several different metrics have been used to measure the fitness of a plant, mostly aiming at evaluating the survivability of the plant, though some use humans to aesthetically evolve the plants.
Multiple genetic operators have been used to modify the the L-system genes in order to evolve them, and they are often specifically designed for L-systems in order to not invalidate the syntax, and often only parts of the L-systems can be modified.

This thesis introduces a concept called Distribution-based Grammar Evolution of L-systems (DGEL), based on the use of GE to evolve L-systems. %cite
It follows the traditional GE method, %cite
but introduces a grammar distribution to control which parts of the grammar that should be weighted the most.
Because varied plants are required, DGEL does not restrict the grammar and modification of the L-system as much as the previous research does.
But with a less restricting grammar, the search space becomes significantly larger.
Thus the distribution will help focus the search space without actually restricting it.
Additionally, different distributions may produce particular plants and thus multiple distributions could be used together to generated varied plants.

At the heart of DGEL is the model of the L-system, which consists of a chromosome, a grammar description and a grammar distribution.
This novel underlying representation of the L-system adapts GE to work with a distribution, and is also applicable to problems not involving L-systems, such as evolution of programs defined by programming languages.
DGEL also consists of two processes that generate parts of the model.
Simulated Annealing (SA) is used to find grammar distributions that GE will used to evolve chromosomes that become aesthetically pleasing plants.
To measure how aesthetically pleasing the plants are, metrics are adapted from the literature and some new are created.
They focus on the structure and shape of the plant, the foliage on it, and physically impossible situations.
To simplify the problem, only branches were allowed in the L-systems, and branch widths and leaves were added as heuristics to the generated plant so that it would not look dead.

DGEL was evaluated in three parts: GE, SA/distribution and fitness.
Each of GE's parameters, except for the tournament size, were found to be improving the performance of it with larger values.
The crossover operator rate interestingly improved the performance until it went from 0.5 to 1.0 where it worsened the performance by a large margin.
GE was shown to be an improvement over random brute-force search when both generated the same amount of individuals.
It was additionally several times faster than random brut-force search, likely caused by the tournament sampling not evaluating each individual.

The SA's progress during one run was analysed and it was found that it was able to improve the distribution from a score of 0 to around 0.6.
It also had mostly good locality, but some moves had the worst possible locality.
The SA-optimized grammar distribution was found to generated plants with higher scores than a uniform distribution, and could thus additionally improve the efficiency of GE.
The grammar distribution and an L-system generated by it was studied in-depth to see what effect it could have.
It was found that the grammar distribution had an restricting effect on the L-system, making the produced plants possibly have similarities with each other.
%more...


\section{Future Work}
At the same time DGEL has weaknesses that need to be resolved to make it better in all areas.
Especially the fitness function needs to be improved as it is the most central part of generating ``aesthetically pleasing plants''.
