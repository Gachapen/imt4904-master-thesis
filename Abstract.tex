\chapter*{Abstract}

Plants can be cross-bred to create offspring with features from both parents, but not all combinations will work.
A virtual world could remove this limitation by allowing all combinations of plants to be bred into new offspring with features from both parents.
This could allow users to combine their favorite plants into new species never seen in the real world, thus allowing for entertaining and interesting games to be created around the concept.
The problem is that if the offspring becomes a warped and messy version of both parents, the user may be displeased.
Thus it is important to keep distinct features from both parents in the offspring.

Virtual content can be generated using procedural content generation (PCG).
More specifically, plants can be generated using L-systems.
L-systems are parallel rewrite systems consisting of a grammar than can be expanded and then visualized as 3D modelled plants.
There are multiple types of L-systems appropriate for different situations, and genetic algorithms have been applied to evolve L-systems into varied plants.
This can be used to solve the above problem.

This thesis will do a literature review of L-system representations and what features they consist of; it will propose an algorithm that can combine two parents into an offspring with distinct features from both parents; it will produce a software implementation of the said algorithm; and it will do an analysis of how well users perceive the offspring compared to their parents.

\hypersetup{pageanchor=false}