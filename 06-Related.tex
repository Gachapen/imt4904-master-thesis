\chapter{Related Work}
Xueqiang and Hong developed a generalized L-system method, called GLSM, that better follows botanical principles and increases the expressive ability of L-systems~\cite{Xueqiang2012}.
They emphasize the importance of a stochastic property in the L-systems to reflect botanical principles.

Wensheng and Xinyan model grass with stochastic L-systems and trees with parametric L-systems~\cite{Wensheng2010}.
For the tree models, they create one L-system where four parameters determine the branching angles and branch lengths.
They show how two different set of parameter in the same L-system can generate two different looking trees, and how adding stochastic multiples to the parameters can generate more naturally looking trees.
This is different from stochastic L-systems, because here they multiply stochastic values with parameters in the productions, while stochastic L-systems select productions stochastically.

Ashlock et al.\ use an evolutionary algorithm to evolve both the grammar and the interpreter parameters at the same time~\cite{Ashlock2006}.
This resulted in a larger variety of generated plants.
They pack the grammar and interpreter parameters as real parameters, which made it possible to solve it as a real parameter optimization problem.

Corchado et al.\ use genetic algorithms to grow plants over time, as they argue that L-systems alone inefficiently simulate growth and change over time~\cite{Corchado2009}.
They define the genotype as a compact string representation of a production rule, and the phenotype as the visualized plant.
The contact string representation consists of the predecessor and the successor connected by the ``='' character.
E.g.\ the production $F\rightarrow FF$ would be the string ``F=FF''.
Crossover is performed by selecting two strings based on their fitness and swapping two substrings between the two strings.
They mutate the string in two different ways: either by randomly changing a character in the string with another, or randomly changing a substring in the string with another.
The crossover and mutation needs to ensure that the string is still syntactically correct.
This mainly involves making sure the branching brackets always have a matching start and end bracket.
To solve this, they use Mock's solution~\cite{Mock1998}, where they only select strings with a balanced number of start and end brackets, which may also be zero brackets.

Beaumont and Stepney use Grammatical Evolution (GE) on L-systems to evolve them~\cite{Beaumont2009}.
GE is ``an evolutionary algorithm (EA) that can evolve complete programs in an arbitrary language using a variable-length binary string''~\cite{Oneil2003}.
Beaumont and Stepney do crossover by selecting random parents and selecting a random crossover point.
The crossover point is different for each parent, to allow variation in genome length in the children.
Elitism is used to allow the most fit genomes to pass to the next generation without being mutated.
They find that elitism is a significant advantage and that certain alterations to the grammar have a significant effect.
