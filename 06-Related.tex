\chapter{Related Work}
Xueqiang and Hong developed a generalized L-system method, called GLSM, that better follows botanical principles and increases the expressive ability of L-systems~\cite{2012Xueqiang}.
They emphasize the importance of a stochastic property in the L-systems to reflect botanical principles.

Wensheng and Xinyan model grass with stochastic L-systems and trees with parametric L-systems~\cite{2010Wensheng}.
For the tree models, they create one L-system where four parameters determine the branching angles and branch lengths.
They show how two different set of parameter in the same L-system can generate two different looking trees, and how adding stochastic multiples to the parameters can generate more naturally looking trees.
This is different from stochastic L-systems, because here they multiply stochastic values with parameters in the productions, while stochastic L-systems select productions stochastically.

%Niklas pioneered artificial evolution of L-systems in 1986. % Can't find copy of paper. Only have one source from another paper. Remove this?
Jacob argues that manually inferring an L-system from a real plant seen in nature is a difficult and tedious task involving several steps~\cite{1994Jacob}.
Some of these steps can be automated using evolutionary algorithms, specifically defining L-systems, comparing the generated plant with the real plant and correcting the L-system based on how it matched.
Based on this, Jacob defines three parts of evolutionary L-system inference: generation functions that create L-systems within some constraints, evaluation functions that measure the fitness of the interpreted L-system, and modification and selection functions that allow editing of L-systems with evolutionary techniques.

To evolve the L-systems, Jacob uses Genetic Programming (GP) techniques introduced by Koza~\cite{1992Koza}.
The difference is that Jacob uses higher-order building blocks for generation and modification of expressions.
A pool of expression patterns is defined that contains several expressions that each are represented in a tree structure.
Expressions can be combined using pattern matching, such that if a leaf node of an expression matches a root node of another expression, they can be combined.
Thus, an L-systems may be built by plugging together expressions in various ways, and the resulting L-system is always guaranteed to be valid.

Jacob defines multiple genetic operators, such as mutation and crossover.
Mutation replaces a subterm of an expression with another equivalent subexpression.
Crossover exchanges to subexpressions between two parents.
Both operators use pattern matching to find subexpressions in the L-system that can be modified.
Multiple patterns may be defined for each operator and are ranked such that if there is a match with multiple patterns, the one with the highest rank is selected.

The technique described by Jacob is shown to evolve a 3D L-system based on requirements that it should have a complex structure and leaves within a certain cubic boundary.
Jacob took this method further and showed that it could evolve L-systems that looked more like plants with certain requirements~\cite{1995Jacob,1996Jacob,1996Jacob-2}.

Mock continues Jacob's work, but works directly with L-system strings instead of expressions~\cite{1998Mock}.
It is thus more of a Genetic Algorithm (GA) technique rather than a GP technique.
The L-systems used are simple 2D interpreted L-systems that only have one node rewriting rule with branching.
The successor of this single rule is used as the genetic representation in the genetic algorithm.
Two genetic operators are used: crossover and mutation.
Crossover finds a random substring in each parent with equal number of branching brackets and exchanges them.
Mutation also finds a random substring the same way, but replaces it with a new randomly generated string instead.
To select individuals from the population that should be used for the next population, both humans and a fitness function have been used independently.
Humans manually assigned fitness to the phenotypes and selected one individual to keep.
When using humans, the generated L-systems started as weed-like or random doodles and ended up with simple 2D plants.
The fitness used was a simple ``tall and wide plants preferred'' function that ended up generating objects that looked less like pants than the human-selected samples.

Ashlock et al.\ use an evolutionary algorithm to evolve both the grammar and the interpreter parameters at the same time~\cite{2006Ashlock}.
This resulted in a larger variety of generated plants.
They pack the grammar and interpreter parameters as real parameters, which made it possible to solve it as a real parameter optimization problem.

Corchado et al.\ use genetic algorithms to grow plants over time, as they argue that L-systems alone inefficiently simulate growth and change over time~\cite{2009Corchado}.
They define the genotype as a compact string representation of a production rule, and the phenotype as the visualized plant.
The contact string representation consists of the predecessor and the successor connected by the ``='' character.
E.g.\ the production $F\rightarrow FF$ would be the string ``F=FF''.
Crossover is performed by selecting two strings based on their fitness and swapping two substrings between the two strings.
They mutate the string in two different ways: either by randomly changing a character in the string with another, or randomly changing a substring in the string with another.
The crossover and mutation needs to ensure that the string is still syntactically correct.
This mainly involves making sure the branching brackets always have a matching start and end bracket.
To solve this, they use Mock's solution~\cite{1998Mock}, where they only select strings with a balanced number of start and end brackets, which may also be zero brackets.

Beaumont and Stepney use Grammatical Evolution (GE) on L-systems to evolve them~\cite{2009Beaumont}.
GE is ``an evolutionary algorithm (EA) that can evolve complete programs in an arbitrary language using a variable-length binary string''~\cite{2003Oneil}.
Beaumont and Stepney do crossover by selecting random parents and selecting a random crossover point.
The crossover point is different for each parent, to allow variation in genome length in the children.
Elitism is used to allow the most fit genomes to pass to the next generation without being mutated.
They find that elitism is a significant advantage and that certain alterations to the grammar have a significant effect.
