\chapter{Plant Representation With L-system}

\section{L-systems}
L-systems was introduced by Lindenmayer in 1968 as a ``theoretical framework for studying the development of simple multicellular organisms''~\cite{Prusinkiewicz2012}, and then later applied to model plants.
An L-system consists of an alphabet, an axiom and a set of production rules.
It is a parallel rewrite system where each letter in the axiom word is rewritten independently based on the production rules.
This happens iteratively, where a new word replaces the previous word each iteration.
A way to draw the plant based on the L-System has to be applied to visualize it.
Prusinkiewicz and Lindemayer summarize multiple research papers into a comprehensive book about L-systems~\cite{Prusinkiewicz2012}.
For a more comprehensive and in-depth explanation of L-systems, refer to this book.

There are also multiple types of L-systems, including discrete, stochastic, context sensitive and parametric~\cite{Prusinkiewicz2012}.
All of these can be combined into one L-system.
A stochastic, context sensitive and parametric L-system (parametric S2L-System) can model all of the other types of L-System by having 100\% probabilities (discrete), no contexts or no parameters, and is thus the most flexible representation.
To be able to generate realistic looking plants with variations per instance, a parametric S2L-system is required.

Discrete context-free L-systems (D0L-systems) are the simplest L-systems, and can be created using edge rewriting, node rewriting, or both~\cite{Prusinkiewicz2012}.
In these systems, there exists one production rule per letter in the alphabet.
By default the productions will produce the same letter as the input (identity).

Stochastic L-systems add a randomness to the generation of the plants.
Each letter in the alphabet may have multiple productions, each with a probability of being selected~\cite{Prusinkiewicz2012}.
This may simulate how different instances of a plant species may look slightly different, thus making the plants look less synthetic when seen together with other plants of the same species.

Context sensitive L-systems add a context to the production rule.
The context can be on either left, right or both sides of the letter.
A production will only be used if the context matches the surrounding letters in the word.
With this, signal propagation can be simulated either upwards or downwards through the plant~\cite{Prusinkiewicz2012}, and more complex plants may be generated.

Parametric L-systems adds parameters to the letters in the words, and conditial rules~\cite{Prusinkiewicz2012}.
The main benefit of using parametric L-systems, is that it can work with rational numbers rather than only integers.
For example, with non-parametric L-systems, extension of a segment can be modeled with the rule $F\rightarrow FF$, or any number of ``F'' in the successor.
In any case, it will only be possible to grow the segment with a multiple of the previous.
For some plants, this is enough, but to model a larger variety of plants, rational numbers are required.
With parametric L-systems, the same rule would be $F(l)\rightarrow F(l*2)$, where $l$ is the length of the segment.
But now it is also possible to grow it at another rate, e.g.\ $F(l)\rightarrow F(l*1.2)$.
Prusinkiewicz and Lindenmayer describe more of the benefits of parametric L-systems~\cite{Prusinkiewicz2012}.

Prusinkiewicz describe L-systems as having three levels of model specification: partial L-systems, L-system schemata, and complete L-systems~\cite{Prusinkiewicz2012}.
The three different levels go from more abstract to more concrete, where partial L-systems specify which structures may result into which new structures (e.g.\ bud becomes a flower), schemata specifies when the different structure switches happen (e.g.\ when the bud becomes a flower), and complete systems specify the geometry of the plant to be visualized (e.g.\ how the bud and the flower should look).
Different methods exist for each level, and which method to use will depend on the type of plant that should be generated.

L-system schemata is of particular interest because there exists multiple methods to use.
The event of a structure resulting into a new structure is called a ``developmental switch''~\cite{Prusinkiewicz2012}.
The timing of the switches need to be controlled by a control mechanism in the system.
There are two classes of these mechanisms: lineage and interaction.
Lineage mechanisms are transferring information from a module to a descendant, while interaction mechanisms exchange information between cells.
Prusinkiewicz and Lindenmayer descibe some of the mechanisms used.
The stochastic mechanism uses a stochastic L-system to apply probabilities to developmental switches.
A table L-system has multiple tables with different rules that can be applied depending on some external factor.
For example, one table can represent summer and another can represent winter, making different switches happen at differnet seasons.
The delay mechanism can delay the developmental switch by a specified number of iterations, for example making a flower bud bloom after a certain amount of iterations.
Accumulation of components uses parametric L-systems to accumulate some parameter until it reaches a threshold causing the switch to happen.
Parametric L-systems can also be used to control development switches with signals, where a signal travels either upwards or downwards through the plant.~\cite{Prusinkiewicz2012}

A popular way to render complete L-systems is the turtle interpretation, where a cursor (the turtle) follows instructions to draw lines, and the alphabet is the set of instructions to use~\cite{Prusinkiewicz2012}.
Turtle interpretation is in its original form simple and can only draw lines on a 2D image, but it has been extended to draw realistic looking plants in 3D~\cite{Prusinkiewicz1988}.
To use the turtle interpretation, extra parameters describing how much the L-system should be expanded and how it should be drawn has to be defined~\cite{Prusinkiewicz2012}.
For example, the number of rewrite iterations, the branching angle, the segment length, and the segment width have to be specified.
This may depend on the plant to be rendered, as some plants might require a more complex set of drawing instructions.

\section{Features in L-systems}