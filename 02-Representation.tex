\chapter{Plant Representation With L-system}

\section{L-systems}
L-systems was introduced by Lindenmayer in 1968 as a ``theoretical framework for studying the development of simple multicellular organisms''~\cite{Prusinkiewicz2012}, and then later applied to model plants.
An L-system consists of an alphabet, an axiom and a set of production rules.
It is a parallel rewrite system where each letter in the axiom word is rewritten independently based on the production rules.
This happens iteratively, where a new word replaces the previous word each iteration.
A way to draw the plant based on the L-System has to be applied to visualize it.
Prusinkiewicz and Lindemayer summarize multiple research papers into a comprehensive book about L-systems~\cite{Prusinkiewicz2012}.
For a more comprehensive and in-depth explanation of L-systems, refer to this book.

There are also multiple types of L-systems, including discrete, stochastic, context sensitive and parametric~\cite{Prusinkiewicz2012}.
All of these can be combined into one L-system.
A stochastic, context sensitive and parametric L-system (parametric S2L-System) can model all of the other types of L-System by having 100\% probabilities (discrete), no contexts or no parameters, and is thus the most flexible representation.
To be able to generate realistic looking plants with variations per instance, a parametric S2L-system is required.

Discrete context-free L-systems (D0L-systems) are the simplest L-systems, and can be created using edge rewriting, node rewriting, or both~\cite{Prusinkiewicz2012}.
In these systems, there exists one production rule per letter in the alphabet.
By default the productions will produce the same letter as the input (identity).

Stochastic L-systems add a randomness to the generation of the plants.
Each letter in the alphabet may have multiple productions, each with a probability of being selected~\cite{Prusinkiewicz2012}.
This may simulate how different instances of a plant species may look slightly different, thus making the plants look less synthetic when seen together with other plants of the same species.

Context sensitive L-systems add a context to the production rule.
The context can be on either left, right or both sides of the letter.
A production will only be used if the context matches the surrounding letters in the word.
With this, signal propagation can be simulated either upwards or downwards through the plant~\cite{Prusinkiewicz2012}, and more complex plants may be generated.

Parametric L-systems adds parameters to the letters in the words, and conditial rules~\cite{Prusinkiewicz2012}.
The main benefit of using parametric L-systems, is that it can work with rational numbers rather than only integers.
For example, with non-parametric L-systems, extension of a segment can be modeled with the rule $F\rightarrow FF$, or any number of ``F'' in the successor.
In any case, it will only be possible to grow the segment with a multiple of the previous.
For some plants, this is enough, but to model a larger variety of plants, rational numbers are required.
With parametric L-systems, the same rule would be $F(l)\rightarrow F(l*2)$, where $l$ is the length of the segment.
But now it is also possible to grow it at another rate, e.g.\ $F(l)\rightarrow F(l*1.2)$.
Prusinkiewicz and Lindenmayer describe more of the benefits of parametric L-systems~\cite{Prusinkiewicz2012}.

Prusinkiewicz describe L-systems as having three levels of model specification: partial L-systems, L-system schemata, and complete L-systems~\cite{Prusinkiewicz2012}.
The three different levels go from more abstract to more concrete, where partial L-systems specify which structures may result into which new structures (e.g.\ bud becomes a flower), schemata specifies when the different structure switches happen (e.g.\ when the bud becomes a flower), and complete systems specify the geometry of the plant to be visualized (e.g.\ how the bud and the flower should look).
Different methods exist for each level, and which method to use will depend on the type of plant that should be generated.

L-system schemata is of particular interest because there exists multiple methods to use.
The event of a structure resulting into a new structure is called a ``developmental switch''~\cite{Prusinkiewicz2012}.
The timing of the switches need to be controlled by a control mechanism in the system.
There are two classes of these mechanisms: lineage and interaction.
Lineage mechanisms are transferring information from a module to a descendant, while interaction mechanisms exchange information between cells.
Prusinkiewicz and Lindenmayer descibe some of the mechanisms used.
The stochastic mechanism uses a stochastic L-system to apply probabilities to developmental switches.
A table L-system has multiple tables with different rules that can be applied depending on some external factor.
For example, one table can represent summer and another can represent winter, making different switches happen at differnet seasons.
The delay mechanism can delay the developmental switch by a specified number of iterations, for example making a flower bud bloom after a certain amount of iterations.
Accumulation of components uses parametric L-systems to accumulate some parameter until it reaches a threshold causing the switch to happen.
Parametric L-systems can also be used to control development switches with signals, where a signal travels either upwards or downwards through the plant.~\cite{Prusinkiewicz2012}

A popular way to render complete L-systems is the turtle interpretation, where a cursor (the turtle) follows instructions to draw lines, and the alphabet is the set of instructions to use~\cite{Prusinkiewicz2012}.
Turtle interpretation is in its original form simple and can only draw lines on a 2D image, but it has been extended to draw realistic looking plants in 3D~\cite{Prusinkiewicz1988}.
To use the turtle interpretation, extra parameters describing how much the L-system should be expanded and how it should be drawn has to be defined~\cite{Prusinkiewicz2012}.
For example, the number of rewrite iterations, the branching angle, the segment length, and the segment width have to be specified.
This may depend on the plant to be rendered, as some plants might require a more complex set of drawing instructions.

\section{Genetically Evolving L-systems}
Genetic evolution involves crossing genes from parents to create an offspring.
Thus, the research field of L-system genetic evolution is a good starting point for finding out how L-systems can be combined into offspring with distinct features from both parents, potentially providing a basis for answering \textit{RQ2} and \textit{RQ3}.
A set of research papers published between 1995 and 2013 was studied to see what different techniques have been used.

When evolving objects, an evolutionary algorithm (EA) is often used.
EA is composed of a series of steps that will evolve a population over multipe generations based on some criteria.
It begins by generating an initial population of individuals.
Then it evaluates the individuals with a fitness function.
This is then used to select the best individuals to be used for reproduction.
The individuals are then used to create offspring for a new population by using crossover and mutation.
Finally, the new population is evaluated and the steps repeat.
In the context of L-systems, the \textit{object} will be replaced with L-system.

To get a better overview of the EA techniques used for L-systems, the EA has been split into five parts: model, representation, generation, selection, and genetic operators.
The \textit{model} and \textit{representation} are related and sometimes almost equal.
The difference is that the model represents the plant is modelled, while the representation may be a small part of the model or the model represented in a different format.
\textit{Generation} is how the representations are generated, e.g.\ for the initial population, or when replacing parts of an L-system with a new part.
\textit{Selection} is the part of selecting individuals to be used for creating the next generation.
\textit{Genetic operators} are used on the selected individuals to modify the representation by either doing crossover between parents, mutating an offspring, or both.

% From here and below should be moved to Related Work.
%Niklas pioneered artificial evolution of L-systems in 1986. % Can't find copy of paper. Only have one source from another paper. Remove this?
Jacob argues that manually inferring an L-system from a real plant seen in nature is a difficult and tedious task involving several steps~\cite{Jacob1998}.
Some of these steps can be automated using evolutionary algorithms, specifically defining L-systems, comparing the generated plant with the real plant and correcting the L-system based on how it matched.
Based on this, Jacob defines three parts of evolutionary L-system inference: generation functions that create L-systems within some constraints, evaluation functions that measure the fitness of the interpreted L-system, and modification and selection functions that allow editing of L-systems with evolutionary techniques.

To evolve the L-systems, Jacob uses Genetic Programming (GP) techniques introduced by Koza~\cite{Koza1993}.
The difference is that Jacob uses higher-order building blocks for generation and modification of expressions.
A pool of expression patterns is defined that contains several expressions that each are represented in a tree structure.
Expressions can be combined using pattern matching, such that if a leaf node of an expression matches a root node of another expression, they can be combined.
Thus, an L-systems may be built by plugging together expressions in various ways, and the resulting L-system is always guaranteed to be valid.

Jacob defines multiple genetic operators, such as mutation and crossover.
Mutation replaces a subterm of an expression with another equivalent subexpression.
Crossover exchanges to subexpressions between two parents.
Both operators use pattern matching to find subexpressions in the L-system that can be modified.
Multiple patterns may be defined for each operator and are ranked such that if there is a match with multiple patterns, the one with the highest rank is selected.

The technique described by Jacob is shown to evolve a 3D L-system based on requirements that it should have a complex structure and leaves within a certain cubic boundary.
Jacob took this method further and showed that it could evolve L-systems that looked more like plants with certain requirements~\cite{Jacob1995,Jacob1996,Jacob1996-2}.

Mock continues Jacob's work, but works directly with L-system strings instead of expressions~\cite{Mock1998}.
It is thus more of a Genetic Algorithm (GA) technique rather than a GP technique.
The L-systems used are simple 2D interpreted L-systems that only have one node rewriting rule with branching.
The successor of this single rule is used as the genetic representation in the genetic algorithm.
Two genetic operators are used: crossover and mutation.
Crossover finds a random substring in each parent with equal number of branching brackets and exchanges them.
Mutation also finds a random substring the same way, but replaces it with a new randomly generated string instead.
To select individuals from the population that should be used for the next population, both humans and a fitness function have been used independently.
Humans manually assigned fitness to the phenotypes and selected one individual to keep.
When using humans, the generated L-systems started as weed-like or random doodles and ended up with simple 2D plants.
The fitness used was a simple ``tall and wide plants preferred'' function that ended up generating objects that looked less like pants than the human-selected samples.

\section{Features in L-systems}
